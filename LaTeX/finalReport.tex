\documentclass[11pt]{article} 
\usepackage{geometry}
\usepackage[utf8]{inputenc}
\geometry{a4paper}
\usepackage{graphicx}
\usepackage[table,xcdraw]{xcolor}
\graphicspath{{images/}}

\begin{document}

%----------------------------------------------------------------------------------------
%	TITLE PAGE
%----------------------------------------------------------------------------------------

\begin{titlepage}

\newcommand{\HRule}{\rule{\linewidth}{0.5mm}}
\center
\textsc{\LARGE King's College London}\\[1.5cm]
\textsc{\Large Traffic Simulator}\\[0.5cm]
\textsc{\large Group Project}\\[0.5cm]
\HRule \\[0.4cm]
{ \huge \bfseries Team Diversity}\\[0.4cm]
\HRule \\[1.5cm]

\begin{minipage}{0.4\textwidth} \large
\begin{center}
\emph{Members:}\\
Balázs Kiss \\
Eddy Mukasa \\
Pongsakorn Riyamongkol \\
Snorri Hannesson\\
Yukolthep Visessmit 
\end{center}
\end{minipage}
\\[2cm]

\includegraphics{KingsLogo}\\[1cm] 

{\large \today}\\[3cm]

\vfill

\end{titlepage}

%----------------------------------------------------------------------------------------
%	Table of contents
%----------------------------------------------------------------------------------------

\tableofcontents

\newpage
%----------------------------------------------------------------------------------------
%	Introduction: Describe the context for the work and the problem you are addressing. Briefly summarise what you achieved in the project.
%----------------------------------------------------------------------------------------

\section{Introduction}
	\subsection{Background}
	\indent\indent Over the past few decades, the world’s population has been continuously increasing which is becoming issues in some countries. This is resulted in overwhelming traffic in the most major cities around the world such as London, Bangkok, and New York. Those cities are looking for the way to solve traffic congestion problem. There are many theories and methods to handle this issues. For example, in Bangkok, it is used priority lane in the peak time hour to relive traffic congestion problem.  Above all, any theories and methods which is applied for solving this problem may need to use together with good traffic management policies. In this paper, it will suggest to use traffic simulator program to test the traffic management policies before these policies can be implemented in the real world.This traffic simulator are supposed to be abstract models of the real world, if a traffic management policy works well on the simulator. This probably may work in the real world.  Therefore, this is the reason why the traffic simulator program is generated. 
	
	\subsection{Descriptions}
	\indent\indent The traffic simulator is an abstract model of actual real world traffic. The simulator will have both cars and buses. There are three type of drivers’ behaviour— cautious, reckless or normal. In this simulator, there are two different traffic management policies witch are supposed to relief traffic congestion issues. Those two policies are fixed time policy and congestion control policy. This policies can be compared by average time each vehicle is in the system. Moreover, an emergency strategy will be implemented to see how the system reacts when an ambulance is injected to the system.\\
	\indent The simulator will be programmed in Java programming language. The simulator will have a graphical user interface (GUI). The GUI is created with the help of JavaFX software platform. The rational for using a GUI: 1. Better visualisation and understanding of code during development, i.e. actually seeing what is happening when programming collision detection. 2. When the final product is ready users can see the road system and the cars and therefore get a better understanding of how the road system is and how the policies work. Opposed to just get the result log and results of which policy is superior and have no visual understanding of what happened. 

	\subsection{Objectives}
	\indent\indent- To develop a traffic simulator program which has the following structure: two types of vehicles, three types of drivers, functional road system with many roads and lanes, junctions, intersections and traffic lights. \\
	\indent - To compare two different traffic management policies: Fixed Time Policy and Congestion Control Policy. \\
	\indent - To examine how the system reacts in an emergency period  by injecting an ambulance to the simulator. 
	
	\subsection{Scope}	
	\indent\indent - Each road can have multiple lanes, which can be in the same or opposite direction. \\
	\indent - British’s traffic is left-lane oriented. \\
	\indent - The system has only two types of vehicle (cars and buses) \\
	\indent - There are three types of driver behaviour which is  cautious, reckless, and normal. 
	
	\subsection{Methodology} 
	\indent\indent I.	Analysing: requirement \\
	\indent II.	Planing and Organising: schedule and assigned task\\
	\indent III.	Developing: \\
			\indent\indent - Software: used JAVA\\
			\indent\indent - Source code: stored at GitHub\\
	\indent IV.	Evaluating:\\
			\indent\indent - Program: used SWOT analysis\\
			\indent\indent - Team: used SWOT analysis\\
			\indent\indent - Peer Assessment\\
	\indent V. 	Reporting:\\
			\indent\indent - Document: written by LaTex\\
			\indent\indent - Presentation
			
	\subsection{Schedule}
		\indent\indent The development of Traffic Simulator program is divided into three iteration in 10 weeks. In the first iteration (week 1 to week 4), our team has focused on the requirement and how to design. After our plan is committed, we have started to develop traffic simulator programme. Then, we has continued to complete the mandatory function in the second iteration (week 5 to week 7). Moreover, in this period, we have finished traffic management policy function, as a program optional and done the unit testing also. In the last iteration ( week 8 to week 10 ). this traffic simulator program has been launched after finished second unit testing and evaluation. The detail of traffic simulator progress is illustrate in the figure, as followed.      
\begin{center}			
			\includegraphics[scale = 0.35]{Figure01}
\end{center}
	\subsection{Obstacle}
		\indent\indent - Language: \\
		\indent\indent Our member come from different country especially in Thailand, Iceland, Hungary and………….. Those country has their own language which caused to the communication issue. Communication is the most important aspect for team work to understand any function of tasks and behaviour also. \\
		\indent\indent However, in this case, we try to communicate more by using another channel. For example, we use picture helping our member to understand the same concept and point that we need to achieve. Therefore, we are able to get rid of language issues between each other and we do more communication.  \\     
	\indent - Skill and Background:\\
		\indent\indent Our member has different skills and background in programming. There are some members who have experience about software development due to their own job. In addition, this project is not required only programming skill but other related skill also (communicating and reporting etc.) Nevertheless, each member is not able to be perfect skill, as the project aims. \\
		\indent\indent At this point, we know that our member has an unique skill and background. We need to help each other to complete the project task. We will train our member to improve other skill which this person does not have or less. Thus, we can overcome this problem and use this problem to gain more benefit. \\
	\indent - Time: \\
		\indent\indent We have got only 10 weeks for developing traffic simulator which is a large number of requirements. \\
		\indent\indent On the other hands, we have set task schedule for each person in deep detail. This help us to follow up our task and complete the project on time. 


	
\newpage
%----------------------------------------------------------------------------------------
%	Related Work: Describe related work.
%----------------------------------------------------------------------------------------
\section{Related Work}
\indent\indent In today's world a lot of effort is made to make transportation as good as possible. For most countries and cities the road systems is a vital part of it's transportation system. With the ever growing population and increasing purchasing power of the public, good traffic control has never been as urgent. Changes to road systems are hard to make and drivers wouldn't be pleased if many experiments were made on live traffic. That's why traffic simulators play a big role in increasing the quality of the road system. Any change can be simulated and the result of the change can be analysed. There are many challenges that traffic simulation creators face. Trying to predict the behaviour of drivers and the synergistic effects of different factors can have on a driver behaviour is perhaps the most challenging, the goal is to make it as realistic as possible.\\

Many traffic simulators exist as well as many papers and books on that subject. In this module we were given two papers on traffic simulation for inspiration for this project and an insight into this field. Sewall, Wilkie, Merrell and Lin \cite{sewall2010continuum}presented a novel method for the synthesis and animation of realistic traffic flows on large-scale road networks. Their technique is based on a continuum model of traffic flow they extended to correctly handle lane changes and merges, as well as traffic behaviours due to changes in speed limit. They demonstrated how their method can be applied to the animation of many vehicles in a large-scale traffic network at interactive rates and showed that their method can simulate believable traffic flows on publicly available, real-world road data. They furthermore demonstrated the scalability of this technique on many-core systems.\\

Namekawa, F. Ueda, Hioki, Y. Ueda and Satoh \cite{namekawa2005general} spent several years developing a general purpose road-traffic simulation system to analyse road traffic jams. The concept of their system was using the running line model as opposed to fixed road-network information database, which is not effective in their opinion. Their simulator uses the a cell automaton model.


\newpage	
%----------------------------------------------------------------------------------------
%	Requirement and Design: Describe the requirements you set for your project at the beginning and the design you have taken for your project. Focus on why you decided to tackle the problem in the way you did, and what effects that had on the design. You may also wish to mention the impact of team-working on your requirements and design.
%----------------------------------------------------------------------------------------	
\section{Requirement and Design}
	\subsection{Requirement}
	\subsection{Design}
	
\newpage
%----------------------------------------------------------------------------------------
%	Implementation: Describe the most significant implementation details, focussing on those where unusual or detailed solutions were required. Quote code fragments wherenecessary, but remember that the full source code will be included as an appendix. Ex- plain how you tested your software (e.g. unit testing) and the extent to which you tested it. If relevant to your project, explain performance issues and how you tackled them.
%----------------------------------------------------------------------------------------
\section{Implementation}

\newpage
%----------------------------------------------------------------------------------------
%	Team Work: Describe how you worked together, including the tools and processes you used to facilitate group work.
%----------------------------------------------------------------------------------------
\section{Team Work}
We have a physical meeting every Thursday at 10 o'clock. We have a log of all meetings. Other means of collaborations are:
\newline \\
\textbf{Facebook} is used as a communication channel. \\
\textbf{GitHub} is used for version control. \\
\textbf{Trello} is used for project management.

\newpage
%----------------------------------------------------------------------------------------
%	Evaluation: Critically evaluate your project: what worked well, and what didn’t? how did you do relative to your plan? what changes were the result of improved thinking and what changes were forced upon you? how did your team work together? etc. Note that you need to show that you understand the weaknesses in your work as well as its strengths. You may wish to identify relevant future work that could be done on your project.
%----------------------------------------------------------------------------------------
\section{Evaluation}
	\indent\indent At this chapter, we are going to use SWOT analysis to evaluate our team and our work. This is very helpful methodology for analysing and evaluating. Moreover, we will then focus on the current study (what we have done) and future study (what we can do next). Therefore, this evaluation chapter is separated into four parts— our team, our program, current study, and future study, details as followed.
	\subsection{Our Team}
	\indent\indent SWOT analysis for our team is following.
	\begin{center}			
			\includegraphics[scale = 0.4]{Figure02}
	\end{center}
	\begin{itemize}
		\item[1. ]\underline{Strengths}: 
		\begin{itemize}
			\item{} Work Well: Our team member has a good collaboration between each other. This caused us to achieve project goals. In other words, our work has been perfectly done and absolutely completed by good collaboration. Therefore, this is our strength making us successful in our task. 
			\item{} Member Behaviour: It contributes to a good behaviour of our member which help us to be successful in the project. This means that each member has high effort, efficiency and initiative. Moreover, they have positive attitude and enthusiasm. Thus, this is the reason why member behaviour is our advantage point for our team. \\\\\\\\\\
  		\end{itemize}				
		
		\item[2. ] \underline{Weaknesses}:
		\begin{itemize}
			\item{} Skill: This is about various skills of each member having, but team member has got only a different skill. It is very useful when everyone has both various skills and same skill. We think that this is not good for us because everyone has a different skill. This means that some tasks is loaded to only one member in our team. Other member in our team wish to help but they are not able to help. Although this seem to be a weakness of us, the different skill and ability can help us to achieve the objective as well due to a variety of skill. This means a project cannot finished by using only one skills. It require a number of skills to carry on the job. In addition, we have trained each other to fulfil the skill which they have not got. In other words, we have learn a lot of skills by sharing and teaching from our friends. Hence, our team can do the project as well as the professional can do.
  		\end{itemize}					

		\item[3. ] \underline{Opportunities}: 
		\begin{itemize}
			\item{} Scheduler and Planner: It deals with the course schedule which force us to do a project within 10 weeks. This help us to manage and plan in any process dealing with the project such as collaborating and scheduling the project. Therefore, scheduler and planner is an opportunity for our team to be successful. 
			\item{} Course Objective: Our member is very appreciate with the course objective that give an opportunity to work as a team. This is very useful for us to work with other people in a real world because we cannot do a job by ourselves. So, this is our advantage for our team to help us carry on the task.
  		\end{itemize}				

		\item[4. ] \underline{Threats}:
		\begin{itemize}
			\item{} Language: communication is very important for effective team, but our member background does not use English as a mother language. This is result in that we cannot understand the same point or may misunderstand. At this point, we have not any control with the language. However, we are able to communicate together by using another method especially in picture, and example. This means that we can over come this threats. Thus our team may goes well in communication and understanding with this strategy.
  		\end{itemize}					
	\end{itemize}
	\newpage
		\subsection{Our Program}
	\indent\indent SWOT analysis for our traffic simulator program is following
	\begin{center}			
			\includegraphics[scale = 0.4]{Figure03}
	\end{center}
	\begin{itemize}
		\item[1. ]\underline{Strengths}: 
		\begin{itemize}
			\item{} Unique format and coding: This Traffic Simulator Program has developed in JAVA  programming language and for GUI is used JavaFX platform. This means that all code in our program is unique and same format. This leads to be easy to run and comply a program; moreover, it is very useful for coding a program when we combine and migrate function in this program together. So  our program is written in a same format, structure and language.
			\item{} Easy to use: Our program is very simple and easy to use and play. This is because  the user interface is very friendly to everyone. Therefore, this point is our strength for Traffic Simulator program.
			\item{} Easy to develop and implement: As the unique format and coding mentioned before, it is the fundamental principle for developing and implementing the program. Our program is very easy to develop and implement due to the unique format and coding. If the next developer would like to develop/change/implement, this system require only JAVA programming skill. In addition, our program is written in JAVA , so any platform which support JAVA is able to implement and run the program as well. Hence, this is a reason why our program is good at development and implement.   
  		\end{itemize}				

		\item[2. ] \underline{Weaknesses}:
		\begin{itemize}
			\item{} Vehicle: The limitation of the program is about the vehicle which has only cars and buses. It may not apply to the real world because there is possible to has more types of vehicle such as bicycle, truck, van etc. However, our program is a traffic simulator which simulate any factors of the traffic as much as possible. We have done only two types of vehicle (cars and buses). This may help me to easily find some bugs and errors of the program before implement to real world. It is very useful for us to do small amount of vehicle before create and apply many types of vehicle. So that, we can develop more when this two types of vehicle are run smoothly.      
			\item{} Map: It is not used the real map to make reality sense for user, but we have created the complex map to provide the more intersection instead. This means this program may use in the real situation if we have generated real map. If we have more time, we can do the real map. In contrast, we use the simulator map for testing our traffic management policy. Thus, it is not necessary to use the real map because if this policy is good, it should be applied in any maps. 
  		\end{itemize}					

		\item[3. ] \underline{Opportunities}: 
		\begin{itemize}
			\item{} Using standard hosting service: We has been forced to use GitHub by the course. GitHub provides a lots of benefit to our team for doing this project. For instance, GitHub can keep track of change and compare changing quickly. So GitHub is the standard hosting service that we have use to manage and store our program.
			\item{} Using standard document format: It is about the markup language LaTex that we have used to make a document. Latex is a compulsory language for making a report. LaTex is generally accepted that it is relatively stable. In other words, LaTex can compile perfectly with use small resource instead of using Microsoft Word or Pages. It can run in any operating system. In addition, LaTex has excellent support for referencing, content, and mathematics symbol. Therefore, this is very benefit for us to produce a report of the traffic simulator program.
			\item{} Project requirement: the project requirement would like us to develop a simulation engine for testing traffic management policies. This is about vehicle, driver behaviour, traffic light management, and traffic management policy. If we have not got these requirement, as mentioned before, our simulator program will not be good. Hence, this is an opportunity for us to show our ability in the programming, teamwork and related useful skills. 

  		\end{itemize}				

		\item[4. ] \underline{Threats}:
		\begin{itemize}
			\item{} Time: The limitation of time is an obvious threat that we have faced in our program. This is because  we have 10 weeks for developing a simulator. It sounds that we have enough time to do a program, but we may not complete all of the project requirement within 10 weeks. The project requires a number of function especially in the vehicle types. Although this seem to be an obstacle for use, this help us to create a critically plan for complete a task. This help us to look for solutions or methods to run this project on time (not perfect, as we expect). Above all, if we have got time more than 10 weeks, we strongly can finish perfect simulator program than before.  So, this is our threat that we cannot control and difficult to get rid off.
  		\end{itemize}					
	\end{itemize}
	\subsection{Current Study}
	\indent\indent According to our plan and objectives, we have absolutely done all mandatory program as well as traffic management policy and show result, as optional program. We cannot finished emergency strategy in time due to the short period of time. The emergency strategy is an optional for our traffic simulator program. We try to follow our plan as much as possible, but some functions of our program take more time than we expected. This is because those functions are complicated which require specific coding and algorithm. However, we strongly believe  that our program seems to be practical and may apply in the real situation. Moreover, the Traffic Simulator program provides both two traffic management policies (Fixed time and Congestion Control)  and three driver behaviours (normal, cautious, reckless). This is very useful and helpful for the user to take into account about traffic management. Therefore, this program is satisfied because it is completed a mandatory requirement.   
	\subsection{Future Study}
	\indent\indent Our program could be developed in the following area in the near future.\\ 
		\indent\indent - Emergency Strategy\\
		\indent\indent - Types of vehicle: assign more vehicle (i.e. trucks, motorcycle, bicycle, and van etc.)\\
		\indent\indent - Real map and traffic route.\\
		\indent\indent - Adding more traffic management policy. 
    

\newpage	
%----------------------------------------------------------------------------------------
%	Peer Assessment: In a simple table, allocate the 100 ‘points’ you are given to each team member. Valid values range from 0 to 100 inclusive. You may assign decimal values, but the entire points must add up to precisely 100.
%----------------------------------------------------------------------------------------
\section{Peer Assessment}

\indent\indent Robert T. \cite{roberts2006self} states that “The term peer assessment refers to the process of having the learners critically reflect upon, and perhaps suggest grades for, the learning of their peers.” In the other words, peer assessment is a process which student are able to assess their friends based on the criteria. This causes student to provide some feedbacks and evaluate their friends, which may help learning together (University of Reading, \cite{peerAssessment}). Therefore, TeamDiversity is going to use the peer assessment method for grading our member. This will be going to focus on the methodology which is used for assessment following by the criteria. It will be then shown the result and summary of each member in TeamDiversity. 
	\subsection{How do we evaluate our member?}
	
	\indent\indent I. Distribute the assessment form and assessment criteria to our member.\\
	\indent II. In each member, he/she must score himself/herself as well as other group member. For example, if our team has 4 people, it will grade 1 for yourself and 3 for our friends.\\
	\indent III. When each team member finished a marking (your friends and yourself), the authorise member need to add up all scores and calculate the average score.\\
	\indent IV. The authorise member is going to allocate the proportional of marks, and shows the results.     


	\subsection{What is the criteria that we have used?}
	\indent\indent\indent University of Sydney \cite{AssessmentCiteria} has published the assessment criteria and form on the website. TeamDiversity has adapted both documents in the appropriate way for supporting our task. This criteria is going to evaluate ten aspects of member behaviour. In each aspect, we have scored in the range from 0—10. Thus, the total marks of each member will vary from 0—100 inclusive. There will be then illustrated the detail in each aspect of peer assessment criteria, as followed \newline 

\textbf{A. Quantity of Work:}\\
	\indent\indent 0	- not taking part in it, having no prospect of progress/value \\
	\indent\indent 1—2	- doing a particular, not too much but enough\\
	\indent\indent 3—4	- sometimes above standard, generally needs improvement \\
	\indent\indent 5—6	- satisfactory, doing more than requirement \\
	\indent\indent 7—8	- always working hard and consistent \\
	\indent\indent 9—10	- outstanding, always over productivity standards \\

\textbf{B. Quality of Work:}\\
	\indent\indent 0	- not giving sufficient attention, making frequent mistakes\\
	\indent\indent 1—2	- giving attention, making some mistakes\\
	\indent\indent 3—4	- doing well, basically correct\\
	\indent\indent 5—6	- satisfactory, accurate in some aspect\\
	\indent\indent 7—8	- almost accurate in all involving fields\\
	\indent\indent 9—10	- outstanding, perfect work\\

\textbf{C. Communication Skills:}\\
	\indent\indent 0	- having bad manners, not showing respect for other people, not listen\\
	\indent\indent 1—2	- friendly and easy to talk to once know by others\\
	\indent\indent 3—4	- warm and friendly, sociable\\
	\indent\indent 5—6	- showing good manners, kindly, listens and understands\\
	\indent\indent 7—8	- courteous and respectful, best wish\\ 
	\indent\indent 9—10	- Inspiring to others, excellent at listening and understanding\\

\textbf{D. Initiative:}\\
	\indent\indent 0	- acts without plan/purpose\\
	\indent\indent 1—2	- need encouragement to do task\\
	\indent\indent 3—4	- putting in minimal effort to complete task\\
	\indent\indent 5—6	- desire to achieve task/goal\\
	\indent\indent 7—8	- strongly desire to achieve task/goal\\
	\indent\indent 9—10	- beyond duty, high motivation\\

\textbf{E. Efficiency:}\\
	\indent\indent 0	- always delayed\\
	\indent\indent 1—2	- occasionally finished on time\\  
	\indent\indent 3—4	- usually finished on time, having minor errors\\
	\indent\indent 5—6	- always finished on time\\
	\indent\indent 7—8	- absolutely completed, consistent in troubleshooting and solving major problems\\ 
	\indent\indent 9—10	- invariably completed ahead of schedule, showing creativity, making major contributions\\ 

\textbf{F. Personal Relations:}\\
	\indent\indent 0	- very disruptive influence\\
	\indent\indent 1—2	- some friction\\
	\indent\indent 3—4	- no problem, commonly\\
	\indent\indent 5—6	- satisfactory, tuneful, harmonious\\ 
	\indent\indent 7—8	- positive factor\\
	\indent\indent 9—10	- respect by others\\

\textbf{G. Group Meeting Attendance:}\\
	\indent\indent 0	- never attended to meeting, not interest\\
	\indent\indent 1—2	- sometime attended \\
	\indent\indent 3—4	- usually attended, hard to get touch with\\
	\indent\indent 5—6	- attend, normally late\\
	\indent\indent 7—8	- count on to attend\\
	\indent\indent 9—10	- never ever missed a meeting, on time\\

\textbf{H. Attitude and Enthusiasm:}\\
	\indent\indent 0	- low disposition, having no prospect of value, unconcerned \\
	\indent\indent 1—2	- feeling/showing few excitement, blasé\\
	\indent\indent 3—4	- half hearted \\
	\indent\indent 5—6	- positive outward behaviour/bearing\\
	\indent\indent 7—8	- positive attitude and spirited\\
	\indent\indent 9—10	- excitement and eager, inspiring to others, positive thinking and influence\\

\textbf{I. Effort:}\\
	\indent\indent 0	- expects others to carry the load\\
	\indent\indent 1—2	- leave some effort\\
	\indent\indent 3—4	- displays enough endeavour\\
	\indent\indent 5—6	- firm and stable contributions\\
	\indent\indent 7—8	- energetic\\
	\indent\indent 9—10	- self starter, normally beyond duty\\

\textbf{J. Dependability:}\\ 
	\indent\indent 0	- unreliable\\
	\indent\indent 1—2	- unsteady, but slightly dependability \\
	\indent\indent 3—4	- inconsistent, occasionally be\\
	\indent\indent 5—6	- suitable, need some improvement \\
	\indent\indent 7—8	- very trustworthy, responsibility \\
	\indent\indent 9—10	- always responsible, steady influence \\
	\subsection{What is there result of peer assessment?}
	\begin{center}			
			\includegraphics[scale = 0.4]{Figure04}
	\end{center}
	

\newpage	
%----------------------------------------------------------------------------------------
%	References
%----------------------------------------------------------------------------------------
\bibliographystyle{plain}

\bibliography{References}

\appendix % Cue to tell LaTeX that the following 'chapters' are Appendices

% Include the appendices of the thesis as separate files from the Appendices folder
% Uncomment the lines as you write the Appendices

% Appendix A

Appendix
llalalla

lfdsfa



% Old one

\textbf{Project description}


\textbf{Background}
Over the past few decades, the world's population has been continuously increasing. This increase in population has resulted in overwhelming traffic which is becoming a serious problem in most major cities around the world. Cities like London, Bangkok, and New York are faced with serious traffic congestion challenges. To be able to handle these challenges and make the traffic run smoother many different traffic management policies have been tried. Before these policies can be implemented in the real world they are tested on traffic simulators. These simulator are supposed to be abstract models of the real world, so if a traffic management policy works on the simulator it probably works in the real world. In this module we will develop a traffic simulator.


\textbf{Objectives}
\begin{itemize}
\item[•] To develop a traffic simulator program which has the following structure: two types of vehicles, three types of drivers, functional road system with many roads and lanes, junctions, intersections and traffic lights.
\item[•] To compare two different traffic management policies: Fixed Time Policy and Congestion Control Policy.
\item[•] To examine how the system reacts in a time of emergency by injecting an ambulance to the simulator.
\end{itemize}

\textbf{Scale}
\begin{itemize}
\item[•] Each road can have multiple lanes, which can be in the same or opposite direction.
\item[•] As in Britain the traffic is left-lane oriented.
\item[•] The system will have cars and buses.
\item[•] Driver behaviour can be cautious, reckless, or normal.
\end{itemize}

\textbf{Outline}
\begin{description}
\item[Description]:
The traffic simulator is an abstract model of actual real world traffic. The simulator will have both cars and buses. Drivers' behaviour can be cautious, reckless or normal. Two different traffic management policies will be implemented witch are supposed to relief traffic congestion issues. The two policies are fixed time policy and congestion control policy. These policies can be compared by average time each vehicle is in the system. Moreover, an emergency strategy will be implemented to see how the system reacts when an ambulance is injected to the system. 
\newline
The simulator will be programmed in Java programming language. The simulator will have a graphical user interface (GUI). The GUI is created with the help of JavaFX software platform. The rational for using a GUI: 1. Better visualisation and understanding of code during development, i.e. actually seeing what is happening when programming collision detection. 2. When the final product is ready users can see the road system and the cars and therefore get a better understanding of how the road system is and how the policies work. Opposed to just get the result log and results of which policy is superior and have no visual understanding of what happened. 
\item[General Concept]:
	\begin{itemize}
		\item[1. ]\underline{Vehicle Types}: There are two types of vehicles: cars and buses. Cars go faster and have higher acceleration than buses. The shape and size of the bus is bigger than car, so it takes more space on the road.
		
		\item[2. ] \underline{Driver Behaviour}: The drivers can behave in three different ways: cautious, reckless, and normal. Reckless drivers drive faster than normal but the cautious drivers drive slower than normal. Reckless drivers are more prone to change lanes and overtake other vehicles, if a fast vehicle is stuck behind a slow vehicle the fast vehicle will try to overtake the slow vehicle.

		\item[3. ] \underline{Emergency Strategy}: When an ambulance is injected to the system it gets highest priority. That means that cars on the right lane will change to the left lane if possible to free the right lane for the ambulance. Also, the ambulance gets to pass at the intersection first.

		\item[4. ] \underline{Traffic Management Policy}: There are two different traffic management policies; \textit{Fixed Time policy} and \textit{Congestion Control policy}:

		\begin{description}
		\item The \textbf{Fixed Time Policy} will have peak hours (when the traffic density is the highest) and off-peak hours (when the traffic density is normal). On peak hours will the duration of the green light be longer than on off-peak hours.
		\item The \textbf{Congestion Control Policy} will always have the same duration of green light in each direction unless the system detects a congestion. When a congestion is detected in some direction the duration of the green light will be increased in that direction for one time to avoid traffic jam. \\
		\end{description}
		
The two different policies will be compared by average time each vehicle is in the system. Each vehicle will have a timer that starts when it enters the system and gets written to a log when the vehicle exits the system. The average time that a vehicle is in the system is calculated and the the policy that generates lower average time is considered better.
		 		
		
	\end{itemize}
\end{description}

\textbf{Schedule}

The time period is divided into three iterations and then the tasks allocated to the iterations. Each task is either mandatory or optional for the simulator:

\begin{table}[h]
\begin{tabular}{
>{\columncolor[HTML]{9B9B9B}}l 
>{\columncolor[HTML]{32CB00}}l 
>{\columncolor[HTML]{32CB00}}l 
>{\columncolor[HTML]{C0C0C0}}l 
>{\columncolor[HTML]{C0C0C0}}l 
>{\columncolor[HTML]{3166FF}}l 
>{\columncolor[HTML]{C0C0C0}}l }
\cellcolor[HTML]{FFFFFF}                                                                    & \multicolumn{4}{c}{\cellcolor[HTML]{FFFFFF}{\color[HTML]{32CB00} Mandatory}}                                                                                                                                                                                                                                              & \multicolumn{2}{c}{\cellcolor[HTML]{FFFFFF}{\color[HTML]{3166FF} Optional}}                                                                                                \\
\begin{tabular}[c]{@{}l@{}}\textbf{Iteration 1:}\\ 16th of January -\\ 12th of February\end{tabular} & Roads                                                                   & Lanes                                                    & \cellcolor[HTML]{32CB00}\begin{tabular}[c]{@{}l@{}}Multiple\\ vehicles\end{tabular} & \cellcolor[HTML]{32CB00}\begin{tabular}[c]{@{}l@{}}Different\\ driver\\ behaviour\end{tabular} & \cellcolor[HTML]{C0C0C0}                                                            &                                                                                      \\
\begin{tabular}[c]{@{}l@{}}\textbf{Iteration 2:}\\ 13th of February -\\ 5th of March\end{tabular}    & \begin{tabular}[c]{@{}l@{}}Junctions\\ and\\ intersections\end{tabular} & \begin{tabular}[c]{@{}l@{}}Traffic\\ lights\end{tabular} &                                                                                     &                                                                                                & \begin{tabular}[c]{@{}l@{}}Different\\ traffic\\ management\\ policies\end{tabular} &                                                                                      \\
\begin{tabular}[c]{@{}l@{}}\textbf{Iteration 3:}\\ 6th of March -\\ 26th of March\end{tabular}       & \cellcolor[HTML]{C0C0C0}                                                & \cellcolor[HTML]{C0C0C0}                                 &                                                                                     &                                                                                                & \begin{tabular}[c]{@{}l@{}}Time\\ logging\end{tabular}                              & \cellcolor[HTML]{3166FF}\begin{tabular}[c]{@{}l@{}}Emergency\\ Strategy\end{tabular}
\end{tabular}
\end{table}


\textbf{Expectation of Project Outcome}
\begin{itemize}
\item[•] The traffic simulator is developed and implemented.
\item[•] Results showing which traffic management policy is superior or if they are equal.
\item[•] The simulator can be used in case of emergency.
 
\end{itemize}

\textbf{Current progress}
We are currently finished with iteration 1 and we are on schedule. We have implemented the roads, lanes that can go in the same or opposite direction, different vehicles and different drivers' behaviour. However, we are still trying to find the best architecture for the different drivers' behaviour and vehicles. \\
As the name of the team implies; we are a very diversified group. From four different countries in three different continents, each with different programming background and experience. Although some of the members of the team were familiar with git and GitHub, LaTeX and some of the other technical tools we are using, some team members weren't. For everyone in the team to be able to make an contribution we have spent a quite a lot of time discussing those technical tools. \\
The biggest challenge so far is to get everyone on the same page about the project and the tools used for developing and collaboration.


%----------------------------------------------------------------------------------------
%	PART TWO
%----------------------------------------------------------------------------------------

\textbf{Project organisation}
\textbf{Project management}
A slightly adjusted Agile Methodology is used for this project. We have divided the time period for this course into iterations and at the end of each iteration we want to be inching forward towards the final product.

\textbf{Roles}


\textbf{Balázs Kiss:} Lead programmer \\
\textbf{Eddy Mukasa:} Architect \\
\textbf{Yukolthep Visessmit:} Graphical designer \\
\textbf{Pongsakorn N. Riyamongkol:} Project Manager \\
\textbf{Snorri Hannesson:} Tester and Coordinator \\ \newline
Each member of the group has a responsibility to oversee one aspect of the project but is not expected to do all the work defined in his role. I.e. each member should/could do some programming but the lead programmer should oversee the code and make sure nothing is missing and everything is done properly. The same goes for the other roles.

\textbf{Collaboration}
We have physical meeting every Thursday at 10 o'clock. We have a log of all meetings. Other means of collaborations are:
\newline \\
\textbf{Facebook} is used as a communication channel. \\
\textbf{GitHub} is used for version control. \\
\textbf{Trello} is used for project management.


\textbf{Peer Assessment and Self Assessment}
Team members should evaluate their own and other members' performance regarding work done in the group. This can be
done by using an assessment form where every member secretly grades themselves and other members of the team. Then can contributions to
\textbf{GitHub} and \textbf{Trello} be examined for evaluation of members performance.

\textbf{Conflicts}
If we have a conflict during this project the will have democracy; the majority decides. However, if a major conflict arises we will have to contact the instructor to resolve the conflict. We have defined a simple method to resolve conflicts: 
\newline
\textbf{I.} Realise conflict. \\
\textbf{II.} Handle conflict sooner rather than later. \\
\textbf{III.} Find the solution together - democracy. \\
\textbf{IV.} Apologise. \\
\textbf{V.} Appreciate. \\
\end{document}