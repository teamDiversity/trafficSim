\documentclass[11pt]{article} 
\usepackage{geometry}
\usepackage[utf8]{inputenc}
\inputencoding{utf8}
\geometry{a4paper}
\usepackage{graphicx}
\usepackage[table,xcdraw]{xcolor}
\usepackage{pdfpages}
\usepackage{tabu}
\usepackage{longtable}

\graphicspath{{images/}}

\begin{document}

%----------------------------------------------------------------------------------------
%	TITLE PAGE
%----------------------------------------------------------------------------------------

\begin{titlepage}

\newcommand{\HRule}{\rule{\linewidth}{0.5mm}}
\center
\textsc{\LARGE King's College London}\\[1.5cm]
\textsc{\Large Traffic Simulator}\\[0.5cm]
\textsc{\large Group Project}\\[0.5cm]
\HRule \\[0.4cm]
{ \huge \bfseries Team Diversity}\\[0.4cm]
\HRule \\[1.5cm]

\begin{minipage}{0.4\textwidth} \large
\begin{center}
\emph{Members:}\\
Balázs Kiss \\
Eddy Mukasa \\
Yukolthep Visessmit \\
Pongsakorn Riyamongkol \\
Snorri Hannesson
\end{center}
\end{minipage}
\\[2cm]

\includegraphics{KingsLogo}\\[1cm] 

{\large \today}\\[3cm]

\vfill

\end{titlepage}

%----------------------------------------------------------------------------------------
%	Table of contents
%----------------------------------------------------------------------------------------

\tableofcontents

\newpage
%----------------------------------------------------------------------------------------
%	Introduction: Describe the context for the work and the problem you are addressing. Briefly summarise what you achieved in the project.
%----------------------------------------------------------------------------------------

\section{Introduction}
	\subsection{Background}
	\subsection{Descriptions: like a abstract}
	\subsection{Methodology} 
	\subsection{Theory}
	\subsection{Goal}
	\subsection{Scope}
	\subsection{Schedule: Chart + text explain}
	\subsection{Obstacle}
	\subsection{br sum => find word (about program) // no need bc 1.1-1.5 is br sum}
	
\newpage
%----------------------------------------------------------------------------------------
%	Related Work: Describe related work.
%----------------------------------------------------------------------------------------
\section{Related Work}
In today's world a lot of effort is made to make transportation as good as possible. For most countries and cities the road systems is a vital part of it's transportation system. With the ever growing population and increasing purchasing power of the public, good traffic control has never been as urgent. Changes to road systems are hard to make and drivers wouldn't be pleased if many experiments were made on live traffic. That's why traffic simulators play a big role in increasing the quality of the road system. Any change can be simulated and the result of the change can be analysed. There are many challenges that traffic simulation creators face. Trying to predict the behaviour of drivers and the synergistic effects of different factors can have on a driver behaviour is perhaps the most challenging, the goal is to make it as realistic as possible.\\

Many traffic simulators exist as well as many papers and books on that subject. In this module we were given two papers on traffic simulation for inspiration for this project and an insight into this field. Sewall, Wilkie, Merrell and Lin \cite{sewall2010continuum}presented a novel method for the synthesis and animation of realistic traffic flows on large-scale road networks. Their technique is based on a continuum model of traffic flow they extended to correctly handle lane changes and merges, as well as traffic behaviours due to changes in speed limit. They demonstrated how their method can be applied to the animation of many vehicles in a large-scale traffic network at interactive rates and showed that their method can simulate believable traffic flows on publicly available, real-world road data. They furthermore demonstrated the scalability of this technique on many-core systems.\\

Namekawa, F. Ueda, Hioki, Y. Ueda and Satoh \cite{namekawa2005general} spent several years developing a general purpose road-traffic simulation system to analyse road traffic jams. The concept of their system was using the running line model as opposed to fixed road-network information database, which is not effective in their opinion. Their simulator uses the a cell automaton model.


\newpage	
%----------------------------------------------------------------------------------------
%	Requirement and Design: Describe the requirements you set for your project at the beginning and the design you have taken for your project. Focus on why you decided to tackle the problem in the way you did, and what effects that had on the design. You may also wish to mention the impact of team-working on your requirements and design.
%----------------------------------------------------------------------------------------	
\section{Requirement and Design}
	\subsection{Requirement}
	\subsection{Design}
	
\newpage
%----------------------------------------------------------------------------------------
%	Implementation: Describe the most significant implementation details, focussing on those where unusual or detailed solutions were required. Quote code fragments wherenecessary, but remember that the full source code will be included as an appendix. Ex- plain how you tested your software (e.g. unit testing) and the extent to which you tested it. If relevant to your project, explain performance issues and how you tackled them.
%----------------------------------------------------------------------------------------
\section{Implementation}

\newpage
%----------------------------------------------------------------------------------------
%	Team Work: Describe how you worked together, including the tools and processes you used to facilitate group work.
%----------------------------------------------------------------------------------------
\section{Team Work}
We have a physical meeting every Thursday at 10 o'clock. We have a log of all meetings. Other means of collaborations are:
\newline \\
\textbf{Facebook} is used as a communication channel. \\
\textbf{GitHub} is used for version control. \\
\textbf{Trello} is used for project management.

\newpage
%----------------------------------------------------------------------------------------
%	Evaluation: Critically evaluate your project: what worked well, and what didn’t? how did you do relative to your plan? what changes were the result of improved thinking and what changes were forced upon you? how did your team work together? etc. Note that you need to show that you understand the weaknesses in your work as well as its strengths. You may wish to identify relevant future work that could be done on your project.
%----------------------------------------------------------------------------------------
\section{Evaluation}

\newpage	
%----------------------------------------------------------------------------------------
%	Peer Assessment: In a simple table, allocate the 100 ‘points’ you are given to each team member. Valid values range from 0 to 100 inclusive. You may assign decimal values, but the entire points must add up to precisely 100.
%----------------------------------------------------------------------------------------
\section{Peer Assessment}

\indent\indent Robert T. \cite{roberts2006self} states that “The term peer assessment refers to the process of having the learners critically reflect upon, and perhaps suggest grades for, the learning of their peers.” In the other words, peer assessment is a process which student are able to assess their friends based on the criteria. This causes student to provide some feedbacks and evaluate their friends, which may help learning together (University of Reading, 2015). Therefore, TeamDiversity is going to use the peer assessment method for grading our member. This will be going to focus on the methodology which is used for assessment following by the criteria. It will be then shown the result and summary of each member in TeamDiversity. 
	\subsection{How do we evaluate our member?}
	
	\indent\indent I. Distribute the assessment form and assessment criteria to our member.\\
	\indent II. In each member, he/she must score himself/herself as well as other group member. For example, if our team has 4 people, it will grade 1 for yourself and 3 for our friends.\\
	\indent III. When you have completed a score (your friends and yourself), you need to mark total add up all scores and calculate the average score for yourself. \\
	\indent IV. We use only the average score to evaluate our friends and present in this report. 


	\subsection{What is the criteria that we have used?}
		\indent University of Sydney (2015) has published the assessment criteria and form on the website. TeamDiversity has adapted both documents in the appropriate way for supporting our task. This criteria is going to evaluate ten aspects of member behaviour. In each aspect, we have scored in the range from 0—10. Thus, the total marks of each member will vary from 0—100 inclusive. There will be then illustrated the detail in each aspect of peer assessment criteria, as followed \newline 

\textbf{A. Quantity of Work:}\\
	\indent\indent 0	- not taking part in it, having no prospect of progress/value \\
	\indent\indent 1—2	- doing a particular, not too much but enough\\
	\indent\indent 3—4	- sometimes above standard, generally needs improvement \\
	\indent\indent 5—6	- satisfactory, doing more than requirement \\
	\indent\indent 7—8	- always working hard and consistent \\
	\indent\indent 9—10	- outstanding, always over productivity standards \\

\textbf{B. Quality of Work:}\\
	\indent\indent 0	- not giving sufficient attention, making frequent mistakes\\
	\indent\indent 1—2	- giving attention, making some mistakes\\
	\indent\indent 3—4	- doing well, basically correct\\
	\indent\indent 5—6	- satisfactory, accurate in some aspect\\
	\indent\indent 7—8	- almost accurate in all involving fields\\
	\indent\indent 9—10	- outstanding, perfect work\\

\textbf{C. Communication Skills:}\\
	\indent\indent 0	- having bad manners, not showing respect for other people, not listen\\
	\indent\indent 1—2	- friendly and easy to talk to once know by others\\
	\indent\indent 3—4	- warm and friendly, sociable\\
	\indent\indent 5—6	- showing good manners, kindly, listens and understands\\
	\indent\indent 7—8	- courteous and respectful, best wish\\ 
	\indent\indent 9—10	- Inspiring to others, excellent at listening and understanding\\

\textbf{D. Initiative:}\\
	\indent\indent 0	- acts without plan/purpose\\
	\indent\indent 1—2	- need encouragement to do task\\
	\indent\indent 3—4	- putting in minimal effort to complete task\\
	\indent\indent 5—6	- desire to achieve task/goal\\
	\indent\indent 7—8	- strongly desire to achieve task/goal\\
	\indent\indent 9—10	- beyond duty, high motivation\\

\textbf{E. Efficiency:}\\
	\indent\indent 0	- always delayed\\
	\indent\indent 1—2	- occasionally finished on time\\  
	\indent\indent 3—4	- usually finished on time, having minor errors\\
	\indent\indent 5—6	- always finished on time\\
	\indent\indent 7—8	- absolutely completed, consistent in troubleshooting and solving major problems\\ 
	\indent\indent 9—10	- invariably completed ahead of schedule, showing creativity, making major contributions\\ 

\textbf{F. Personal Relations:}\\
	\indent\indent 0	- very disruptive influence\\
	\indent\indent 1—2	- some friction\\
	\indent\indent 3—4	- no problem, commonly\\
	\indent\indent 5—6	- satisfactory, tuneful, harmonious\\ 
	\indent\indent 7—8	- positive factor\\
	\indent\indent 9—10	- respect by others\\

\textbf{G. Group Meeting Attendance:}\\
	\indent\indent 0	- never attended to meeting, not interest\\
	\indent\indent 1—2	- sometime attended \\
	\indent\indent 3—4	- usually attended, hard to get touch with\\
	\indent\indent 5—6	- attend, normally late\\
	\indent\indent 7—8	- count on to attend\\
	\indent\indent 9—10	- never ever missed a meeting, on time\\

\textbf{H. Attitude and Enthusiasm:}\\
	\indent\indent 0	- low disposition, having no prospect of value, unconcerned \\
	\indent\indent 1—2	- feeling/showing few excitement, blasé\\
	\indent\indent 3—4	- half hearted \\
	\indent\indent 5—6	- positive outward behaviour/bearing\\
	\indent\indent 7—8	- positive attitude and spirited\\
	\indent\indent 9—10	- excitement and eager, inspiring to others, positive thinking and influence\\

\textbf{I. Effort:}\\
	\indent\indent 0	- expects others to carry the load\\
	\indent\indent 1—2	- leave some effort\\
	\indent\indent 3—4	- displays enough endeavour\\
	\indent\indent 5—6	- firm and stable contributions\\
	\indent\indent 7—8	- energetic\\
	\indent\indent 9—10	- self starter, normally beyond duty\\

\textbf{J. Dependability:}\\ 
	\indent\indent 0	- unreliable\\
	\indent\indent 1—2	- unsteady, but slightly dependability \\
	\indent\indent 3—4	- inconsistent, occasionally be\\
	\indent\indent 5—6	- suitable, need some improvement \\
	\indent\indent 7—8	- very trustworthy, responsibility \\
	\indent\indent 9—10	- always responsible, steady influence \\
	\subsection{What is there result and summary of peer assessment?}

\newpage	
%----------------------------------------------------------------------------------------
%	References
%----------------------------------------------------------------------------------------
\bibliographystyle{plain}
\bibliography{References}


\newpage	
%----------------------------------------------------------------------------------------
%	Appendices
%----------------------------------------------------------------------------------------
\addtocontents{toc}{\vspace{2em}}

\appendix
% Appendix A

Appendix
llalalla

lfdsfa

% Appendix B: the source code

\section{Source Code}
\label{AppendixB}

\includepdf[pages=-]{Appendices/code/sim.pdf}
\includepdf[pages=-]{Appendices/code/core.pdf}
\includepdf[pages=-]{Appendices/code/tests.pdf}

\newpage	
%----------------------------------------------------------------------------------------
%	This is the initial report:
%----------------------------------------------------------------------------------------

\textbf{Project description}


\textbf{Background}
Over the past few decades, the world's population has been continuously increasing. This increase in population has resulted in overwhelming traffic which is becoming a serious problem in most major cities around the world. Cities like London, Bangkok, and New York are faced with serious traffic congestion challenges. To be able to handle these challenges and make the traffic run smoother many different traffic management policies have been tried. Before these policies can be implemented in the real world they are tested on traffic simulators. These simulator are supposed to be abstract models of the real world, so if a traffic management policy works on the simulator it probably works in the real world. In this module we will develop a traffic simulator.


\textbf{Objectives}
\begin{itemize}
\item[•] To develop a traffic simulator program which has the following structure: two types of vehicles, three types of drivers, functional road system with many roads and lanes, junctions, intersections and traffic lights.
\item[•] To compare two different traffic management policies: Fixed Time Policy and Congestion Control Policy.
\item[•] To examine how the system reacts in a time of emergency by injecting an ambulance to the simulator.
\end{itemize}

\textbf{Scale}
\begin{itemize}
\item[•] Each road can have multiple lanes, which can be in the same or opposite direction.
\item[•] As in Britain the traffic is left-lane oriented.
\item[•] The system will have cars and buses.
\item[•] Driver behaviour can be cautious, reckless, or normal.
\end{itemize}

\textbf{Outline}
\begin{description}
\item[Description]:
The traffic simulator is an abstract model of actual real world traffic. The simulator will have both cars and buses. Drivers' behaviour can be cautious, reckless or normal. Two different traffic management policies will be implemented witch are supposed to relief traffic congestion issues. The two policies are fixed time policy and congestion control policy. These policies can be compared by average time each vehicle is in the system. Moreover, an emergency strategy will be implemented to see how the system reacts when an ambulance is injected to the system. 
\newline
The simulator will be programmed in Java programming language. The simulator will have a graphical user interface (GUI). The GUI is created with the help of JavaFX software platform. The rational for using a GUI: 1. Better visualisation and understanding of code during development, i.e. actually seeing what is happening when programming collision detection. 2. When the final product is ready users can see the road system and the cars and therefore get a better understanding of how the road system is and how the policies work. Opposed to just get the result log and results of which policy is superior and have no visual understanding of what happened. 
\item[General Concept]:
	\begin{itemize}
		\item[1. ]\underline{Vehicle Types}: There are two types of vehicles: cars and buses. Cars go faster and have higher acceleration than buses. The shape and size of the bus is bigger than car, so it takes more space on the road.
		
		\item[2. ] \underline{Driver Behaviour}: The drivers can behave in three different ways: cautious, reckless, and normal. Reckless drivers drive faster than normal but the cautious drivers drive slower than normal. Reckless drivers are more prone to change lanes and overtake other vehicles, if a fast vehicle is stuck behind a slow vehicle the fast vehicle will try to overtake the slow vehicle.

		\item[3. ] \underline{Emergency Strategy}: When an ambulance is injected to the system it gets highest priority. That means that cars on the right lane will change to the left lane if possible to free the right lane for the ambulance. Also, the ambulance gets to pass at the intersection first.

		\item[4. ] \underline{Traffic Management Policy}: There are two different traffic management policies; \textit{Fixed Time policy} and \textit{Congestion Control policy}:

		\begin{description}
		\item The \textbf{Fixed Time Policy} will have peak hours (when the traffic density is the highest) and off-peak hours (when the traffic density is normal). On peak hours will the duration of the green light be longer than on off-peak hours.
		\item The \textbf{Congestion Control Policy} will always have the same duration of green light in each direction unless the system detects a congestion. When a congestion is detected in some direction the duration of the green light will be increased in that direction for one time to avoid traffic jam. \\
		\end{description}
		
The two different policies will be compared by average time each vehicle is in the system. Each vehicle will have a timer that starts when it enters the system and gets written to a log when the vehicle exits the system. The average time that a vehicle is in the system is calculated and the the policy that generates lower average time is considered better.
		 		
		
	\end{itemize}
\end{description}

\textbf{Schedule}

The time period is divided into three iterations and then the tasks allocated to the iterations. Each task is either mandatory or optional for the simulator:

\begin{table}[h]
\begin{tabular}{
>{\columncolor[HTML]{9B9B9B}}l 
>{\columncolor[HTML]{32CB00}}l 
>{\columncolor[HTML]{32CB00}}l 
>{\columncolor[HTML]{C0C0C0}}l 
>{\columncolor[HTML]{C0C0C0}}l 
>{\columncolor[HTML]{3166FF}}l 
>{\columncolor[HTML]{C0C0C0}}l }
\cellcolor[HTML]{FFFFFF}                                                                    & \multicolumn{4}{c}{\cellcolor[HTML]{FFFFFF}{\color[HTML]{32CB00} Mandatory}}                                                                                                                                                                                                                                              & \multicolumn{2}{c}{\cellcolor[HTML]{FFFFFF}{\color[HTML]{3166FF} Optional}}                                                                                                \\
\begin{tabular}[c]{@{}l@{}}\textbf{Iteration 1:}\\ 16th of January -\\ 12th of February\end{tabular} & Roads                                                                   & Lanes                                                    & \cellcolor[HTML]{32CB00}\begin{tabular}[c]{@{}l@{}}Multiple\\ vehicles\end{tabular} & \cellcolor[HTML]{32CB00}\begin{tabular}[c]{@{}l@{}}Different\\ driver\\ behaviour\end{tabular} & \cellcolor[HTML]{C0C0C0}                                                            &                                                                                      \\
\begin{tabular}[c]{@{}l@{}}\textbf{Iteration 2:}\\ 13th of February -\\ 5th of March\end{tabular}    & \begin{tabular}[c]{@{}l@{}}Junctions\\ and\\ intersections\end{tabular} & \begin{tabular}[c]{@{}l@{}}Traffic\\ lights\end{tabular} &                                                                                     &                                                                                                & \begin{tabular}[c]{@{}l@{}}Different\\ traffic\\ management\\ policies\end{tabular} &                                                                                      \\
\begin{tabular}[c]{@{}l@{}}\textbf{Iteration 3:}\\ 6th of March -\\ 26th of March\end{tabular}       & \cellcolor[HTML]{C0C0C0}                                                & \cellcolor[HTML]{C0C0C0}                                 &                                                                                     &                                                                                                & \begin{tabular}[c]{@{}l@{}}Time\\ logging\end{tabular}                              & \cellcolor[HTML]{3166FF}\begin{tabular}[c]{@{}l@{}}Emergency\\ Strategy\end{tabular}
\end{tabular}
\end{table}


\textbf{Expectation of Project Outcome}
\begin{itemize}
\item[•] The traffic simulator is developed and implemented.
\item[•] Results showing which traffic management policy is superior or if they are equal.
\item[•] The simulator can be used in case of emergency.
 
\end{itemize}

\textbf{Current progress}
We are currently finished with iteration 1 and we are on schedule. We have implemented the roads, lanes that can go in the same or opposite direction, different vehicles and different drivers' behaviour. However, we are still trying to find the best architecture for the different drivers' behaviour and vehicles. \\
As the name of the team implies; we are a very diversified group. From four different countries in three different continents, each with different programming background and experience. Although some of the members of the team were familiar with git and GitHub, LaTeX and some of the other technical tools we are using, some team members weren't. For everyone in the team to be able to make an contribution we have spent a quite a lot of time discussing those technical tools. \\
The biggest challenge so far is to get everyone on the same page about the project and the tools used for developing and collaboration.


%----------------------------------------------------------------------------------------
%	PART TWO
%----------------------------------------------------------------------------------------

\textbf{Project organisation}
\textbf{Project management}
A slightly adjusted Agile Methodology is used for this project. We have divided the time period for this course into iterations and at the end of each iteration we want to be inching forward towards the final product.

\textbf{Roles}


\textbf{Balázs Kiss:} Lead programmer \\
\textbf{Eddy Mukasa:} Architect \\
\textbf{Yukolthep Visessmit:} Graphical designer \\
\textbf{Pongsakorn N. Riyamongkol:} Project Manager \\
\textbf{Snorri Hannesson:} Tester and Coordinator \\ \newline
Each member of the group has a responsibility to oversee one aspect of the project but is not expected to do all the work defined in his role. I.e. each member should/could do some programming but the lead programmer should oversee the code and make sure nothing is missing and everything is done properly. The same goes for the other roles.

\textbf{Collaboration}
We have physical meeting every Thursday at 10 o'clock. We have a log of all meetings. Other means of collaborations are:
\newline \\
\textbf{Facebook} is used as a communication channel. \\
\textbf{GitHub} is used for version control. \\
\textbf{Trello} is used for project management.


\textbf{Peer Assessment and Self Assessment}
Team members should evaluate their own and other members' performance regarding work done in the group. This can be
done by using an assessment form where every member secretly grades themselves and other members of the team. Then can contributions to
\textbf{GitHub} and \textbf{Trello} be examined for evaluation of members performance.

\textbf{Conflicts}
If we have a conflict during this project the will have democracy; the majority decides. However, if a major conflict arises we will have to contact the instructor to resolve the conflict. We have defined a simple method to resolve conflicts: 
\newline
\textbf{I.} Realise conflict. \\
\textbf{II.} Handle conflict sooner rather than later. \\
\textbf{III.} Find the solution together - democracy. \\
\textbf{IV.} Apologise. \\
\textbf{V.} Appreciate. \\
\end{document}