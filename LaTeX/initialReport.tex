\documentclass[11pt]{article} 
\usepackage{geometry}
\usepackage[utf8]{inputenc}
\geometry{a4paper}
\usepackage{graphicx}
\usepackage[table,xcdraw]{xcolor}
\graphicspath{{images/}} 
\begin{document}

%----------------------------------------------------------------------------------------
%	TITLE PAGE
%----------------------------------------------------------------------------------------

\begin{titlepage}

\newcommand{\HRule}{\rule{\linewidth}{0.5mm}}
\center
\textsc{\LARGE King's College London}\\[1.5cm]
\textsc{\Large Traffic Simulator}\\[0.5cm]
\textsc{\large Group Project}\\[0.5cm]
\HRule \\[0.4cm]
{ \huge \bfseries Team Diversity}\\[0.4cm]
\HRule \\[1.5cm]

\begin{minipage}{0.4\textwidth} \large
\begin{center}
\emph{Members:}\\
Balázs Kiss \\
Eddy Mukasa \\
Gabb Visessmit \\
Pongsakorn N. Riyamongkol \\
Snorri Hannesson
\end{center}
\end{minipage}
\\[2cm]

\includegraphics{KingsLogo}\\[1cm] 

{\large \today}\\[3cm]

\vfill

\end{titlepage}

%----------------------------------------------------------------------------------------
%	Part One
%----------------------------------------------------------------------------------------

\section{About the project}

\subsection{Background}
In a past of decade, population is seemed to be continuous increased which may not be
predictable. The traffic management is becoming a serious problem a number of cities around the
world due to the crowned people in these areas. The capital city especially in London, Bangkok, and
New York has faced with the traffic congestion challenge. This is because the road is limit but the
demand of road’s user is high. Therefore, the traffic management is the method to relieve this
problem by using the Traffic Simulator.

Traffic Simulator project is a part of coursework in Group Project (7CCSMGPR) of King’s
College London. This course is to provide the development in the software product: planning,
designing, implementing, and reporting. Hence, the Traffic Simulator matches to the aim of Group
Project Course and helps traffic management to resolve traffic challenge.

\subsection{Objectives}
\begin{itemize}
\item[-] Develop a traffic simulator program.
\item[-] To help the traffic management method.
\item[-] To achieve the aim of Group Project (7CCSMGPR) course.
\end{itemize}

\subsection{Objectives}
\begin{itemize}
\item[-] There are two lanes (left and right) in each road. The left lane is normal, but the right lane
is for driver changing lanes or emergency.
\item[-] There has only one-way direction of the vehicle.
\item[-] This programme has only car and bus.
\item[-] This programme is designed for driver behaviour which can only be caution, reckless, and
normal.
\item[-] The scale of the map is ???
\end{itemize}

\subsection{Outline}
\begin{description}
\item[Description]:
The Traffic Simulator is the programme that provides some traffic management policies to relief traffic congestion issues. The policy that forced to the
system is fixed time policy and congestion control policy. This traffic simulator programme has only car and bus which each vehicle driver behavior can be caution, reckless, and normal. Moreover, the emergency strategy is used in the system to make the 	system is reliable. Therefore, the Traffic Simulator will be the one method of the traffic
management policy to reduce the traffic congestion in the main road. 
\item[General Concept]:
	\begin{itemize}
		\item[1. ]\underline{Vehicle Types}: There are two types of 	vehicles—car and bus. This programme is assumed that car go faster and has higher acceleration than bus. The shape and size of the bus is bigger than car, so it takes more space in the road.
		
		\item[2. ] \underline{Driver’s Behaviours}: There are three types of driver’s behaviour-- caution, reckless, and normal. In the simulator, it is judged the behaviour of driver by using speed and acceleration, as illustrated in the table below. Moreover, the driving behaviour of abnormal people may not be the normal people.

\begin{table}[h]
\centering
\begin{tabular}{c|c|c}
\rowcolor[HTML]{C0C0C0} 
Type     & Car (X pixels)                                                                   & Bus  (2X pixels)                                                                 \\ \hline
Cautious & \begin{tabular}[c]{@{}c@{}}Acceleration: Y\\ Top Speed: X\end{tabular}           & \begin{tabular}[c]{@{}c@{}}Acceleration: Y - 10\\ Top Speed: X - 20\end{tabular} \\ \hline
Normal   & \begin{tabular}[c]{@{}c@{}}Acceleration: Y + 10\\ Top Speed: X + 20\end{tabular} & \begin{tabular}[c]{@{}c@{}}Acceleration: Y\\ Top Speed: X\end{tabular}           \\ \hline
Reckless & \begin{tabular}[c]{@{}c@{}}Acceleration: Y + 20\\ Top Speed: X + 40\end{tabular} & \begin{tabular}[c]{@{}c@{}}Acceleration: Y + 10\\ Top Speed: X + 20\end{tabular}
\end{tabular}
\end{table}

		\item[3. ] \underline{Emergency Strategy}: The emergency policy in the simulator is to prioritise ambulances at traffic lights and on the road. When the ambulance is generated, all vehicles on the same road and area will avoid the ambulance. In addition, another vehicle in that area will stop and give the ambulance pass the intersection fist.

		\item[4. ] \underline{Traffic Management Policy}: There are two different traffic management policies—Fixed Time and Congestion Control policy.
		\begin{itemize}
		\item[-] \textit{Fixed Time Policy} is the policy to set the peak and off-peak time during a day. In a peak time period, the green light on the direction to the business centre area or working area will be 2X seconds before changing to red light. And off-peak time period, it will be set the green light time to X seconds before changing.
		\item[-] \textit{Congestion Control Policy} is the policy that automatically changes the duration of the green light due to the congestion of the traffic. The green light time will be longer than other directions, when the congestion occurred.
		
		The two different will be compared by the average time in each vehicle on the system. In each vehicle will have a timer that starts when it enters the system and gets written to a log when the vehicle exists the system. The average time is calculated during this period (vehicle in/out to/from system). The both policies either fixed time or congestion control, which has lower average time in each vehicle, is better than another. 
		\end{itemize} 		
		
	\end{itemize}
\end{description}

\subsection{Methods and Strategies}
\begin{itemize}
\item Software Development: developed by JAVA programming language.
\item Source Code: done by GitHub.
\item Documentation: done by GitHub and created by LaTeX.
\item Management: Trello.
\item Assessment: graded by Burger’s Algorithm and Peer Assessment
\end{itemize}

\subsection{Schedule}

\begin{table}[h]
\centering
\begin{tabular}{|l|l|l|l|l|l|}
\hline
\rowcolor[HTML]{C0C0C0} 
\begin{tabular}[c]{@{}l@{}}Progress\\ and\\ Period\end{tabular} & \begin{tabular}[c]{@{}l@{}}Requirement\\ and Analysis\end{tabular} & \begin{tabular}[c]{@{}l@{}}Design and\\ Code\end{tabular} & \begin{tabular}[c]{@{}l@{}}Integration\\ and\\ Implementation\end{tabular} & Testing                  & Evaluation                             \\ \hline
wk01                                                            & \cellcolor[HTML]{FFCCC9}                                           &                                                           &                                                                            &                          &                                        \\ \hline
wk02                                                            & \cellcolor[HTML]{FFCCC9}                                           & \cellcolor[HTML]{FFCCC9}                                  &                                                                            &                          &                                        \\ \hline
wk03                                                            &                                                                    & \cellcolor[HTML]{FFCCC9}Initial report                    &                                                                            &                          &                                        \\ \hline
wk04                                                            &                                                                    & \cellcolor[HTML]{FFCCC9}Presentation                      &                                                                            &                          &                                        \\ \hline
wk05                                                            &                                                                    & \cellcolor[HTML]{FFCCC9}                                  & \cellcolor[HTML]{FFCCC9}                                                   &                          &                                        \\ \hline
wk06                                                            &                                                                    & \cellcolor[HTML]{FFCCC9}                                  & \cellcolor[HTML]{FFCCC9}                                                   &                          &                                        \\ \hline
wk07                                                            &                                                                    &                                                           & \cellcolor[HTML]{FFCCC9}                                                   & \cellcolor[HTML]{FFCCC9} &                                        \\ \hline
wk08                                                            &                                                                    &                                                           &                                                                            & \cellcolor[HTML]{FFCCC9} &                                        \\ \hline
wk09                                                            &                                                                    &                                                           &                                                                            & \cellcolor[HTML]{FFCCC9} & \cellcolor[HTML]{FFCCC9}Initial report \\ \hline
wk10                                                            &                                                                    &                                                           &                                                                            &                          & \cellcolor[HTML]{FFCCC9}Presentation   \\ \hline
\end{tabular}
\end{table}


\subsection{Expectation of Project Outcome}
\begin{itemize}
\item[-] Traffic Simulator is developed and implemented in the real situation.
\item[-] Traffic Simulator is the one of the method to help the traffic management policy.
\item[-] Traffic Simulator is achieved in any objectives of Group Project (7CCSGPR) programme.
\end{itemize}


%----------------------------------------------------------------------------------------
%	PART TWO
%----------------------------------------------------------------------------------------

\section{About us}
\subsection{Project management}
The Agile Methodology is used for this project. 

\subsection{Roles}
\begin{description}
\item[Balázs Kiss:] Lead programmer
\item[Eddy Mukasa:] Architect
\item[Gabb Visessmit:] Graphical designer
\item[Pongsakorn N. Riyamongkol:] Project Manager
\item[Snorri Hannesson:] Tester and Coordinator
\end{description}

\subsection{Collaboration}
\begin{description}
\item[Facebook:] used facebook group, as a communication channel.
\item[GitHub:] used for collecting all source code and documentation.
\item[Trello:] used for management and follow up any tasks.
\item[Meeting:] set up a meeting in on Thursday 10:00 every week during the project period.
\end{description}

\subsection{Peer Assessment and Self Assessment}
This involves members of our group to evaluate other members regarding the performance of
their tasks. In addition, self-assessment is used to grade yourself in term of duties and
accomplishment tasks. Peer and Self Assessment can be used for both formative and summative
purposes. There are some techniques about peer and self assessment, as followed:
\begin{itemize}
\item[-] Assessment Form
\item[-] Noticed from GitHub and Trello.  
\end{itemize}

\subsection{Conflicts}
If we have conflicts during this project, we also use simply method to resolve it. 
\begin{itemize}
\item[I. ]Realise conflict
\item[II. ] Handle conflict sooner rather than later.
\item[III. ] Find the solution together
\item[IV. ] Apologise
\item[V. ] Appreciate
\end{itemize}



\end{document}